\documentclass{beamer}

\usepackage[utf8]{inputenc}
\usepackage[T1]{fontenc}
\usepackage[frenchb]{babel}

\usepackage{url}

\begin{document}

\title{L'informatique commune en prépa}   
\author{François Fayard} 
\date{\today} 
	
\frame{\titlepage}
  


\frame{
  \frametitle{Les élèves arrivant en prépa}
  \begin{itemize}
    \item L'informatique au lycée
    \item Tous les éléves de prépa suivent une formation informatique.
      Certains éléves de {\sc MP} suivent une option informatique.
  \end{itemize}
}

\frame{
  \frametitle{Le but de la formation}
  \begin{itemize}
    \item Conception d'algorithmes et structures de données
      % Programmation impérative, structurée
    \item Évaluer, contrôler, valider
    \item Communiquer
      % Expliquer à l'écrit un algorithme, documenter un programme
      % Utilisation de bibliothèque logicielles
    \item Contextualiser les problèmes
      % Applications en chimie, physique, mathématiques, SI
  \end{itemize}
}

\frame{
  \frametitle{Le programme de première année}
  \begin{itemize}
    \item Introduction au materiel et au système d'exploitation
    \item Algorithme 1~:
    \begin{itemize}
      \item Notion de type
      \item Programmation impérative
      \item Structures élémentaires de données
      \item Preuve de programmes
      \item Complexité en temps et en espace
        % Algorithmes simples :
        % - Recherche dans une liste
        % - Recherche du maximum
        % - Calcul de la moyenne, variance
        % - Calcul approché d'une intégrale sur un segment
        % - Recherche d'un mot dans une chaîne de caractères
    \end{itemize}
    \item Ingénierie numérique et simulation
      % - Utilisation des bibliothèques Numpy/Scipy ou Scilab
      % - Méthode de dichotomie, de Newton
      % - Résolution d'équations différentielles par des méthodes de type Euler
      % - Résolution de systèmes linéaires par pivot de Gauss
    \item Représentation des entiers et des flottants
    \item Bases de données
      % - Vocabulaire : 
      % - Opérateurs usuels :
      % - Client-serveur
      % - Utilisation d'un client en interface graphique : sqliteman
  \end{itemize}
}

\frame{
  \frametitle{Le programme de seconde année}
  \begin{itemize}
    \item Algorithme 2~:
    \begin{itemize}
      \item Piles et recursivité
      \item Étude de tris~: insertion, quicksort, fusion
    \end{itemize}
    \item Étude pratique de sujets parmi les suivants~:
    \begin{itemize}
      \item Traitement des images
      \item Cryptographie élémentaire
        % - algorithme de Vigenère
      \item Transmission fiable de données
        % - code de Hamming
      \item Algorithmique des graphes
        % - algorithme de Dijkstra
      \item Programmation orientée objets et interfaces graphiques
    \end{itemize}
  \end{itemize}
}

\frame{
  \frametitle{Organisation de l'enseignement}
  \begin{itemize}
    \item En première année~:
    \begin{itemize}
      \item 35 heures de cours magistral
      \item 35 heures de TP sur machine
    \end{itemize}
    \item En seconde année~:
    \begin{itemize}
      \item x heures de cours magistral
      \item x heures de TP sur machine
    \end{itemize}
  \end{itemize}
}

\frame{
  \frametitle{L'évaluation au concours}
  \begin{itemize}
    \item Polytechnique, ENS~:
    \item Concours des Mines~:
    \item Concours Central~:
  \end{itemize}
}


\end{document}